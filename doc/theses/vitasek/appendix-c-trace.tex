\chapter{Experimental Trace}
\label{appendix-trace}

Following is a trace logged from a sample experiment run. It shows all the relevant information related to this instance, any and every piece of information we might be interested in.

To save space, 2-column layout is used. Commentary on the particulars follows right after its end.

\begin{multicols}{2}
\begin{scriptsize}
\begin{verbatim}
CPU info
  Intel(R) Core(TM)2 Quad CPU Q9550 @ 2.83GHz
  Cores: 4
  Clock speed: 2983 MHz
Memory info
  Size: 8192 MB
OS info
  Name: Windows 7
  Version: 6.1
  Architecture: amd64
Java info
  Version: 1.6.0_26
  VM: Java HotSpot(TM) 64-Bit Server VM
GLPK info
  GLPSOL: GLPK LP/MIP Solver 4.34

Configuration:
File name: graph.xml (101599 b)
  Graph representation: 82 vertices, 1101 edges
alpha: 1.0, beta: 1.0

Results:
Total time spent: 7754 ms
Final quality: 0.19951219512195123 (10 AMs)
Highest quality: 0.23463414634146343 (12 AMs)
Construction phase:
  Algorithm: Random
    Time taken: 248 ms / Time since start: 248 ms
    Pool size: 10
    Quality: 0.19975609756097568 (11 AMs)
Improvement phase:
  pass #1:
  Algorithm: RandomRemove, ratio = 0.2
    Time taken: 0 ms / Time since start: 841 ms
    Pool size: 10
    Quality: 0.15878048780487808 (9 AMs)
  pass #2:
  Algorithm: Mutation, ratio = 0.1, limit = 1 s
    Time taken: 1512 ms / Time since start: 2710 ms
    Pool size: 11
    Quality: 0.21975609756097558 (11 AMs)

  <... 7 more passes removed ...>

  pass #10:
  Algorithm: Remove Worst
    Time taken: 80 ms / Time since start: 7676 ms
    Pool size: 12
    Quality: 0.19951219512195123 (10 AMs)
Termination reason: Maximum iterations exceeded.

Time,Quality,AMs
248,0.19975609756097568,11
841,0.15878048780487808,9
2710,0.21975609756097558,11
2927,0.1890243902439024,9
4421,0.23463414634146343,12
4703,0.23463414634146343,12
4896,0.1960975609756098,10
5793,0.23463414634146337,12
5972,0.19951219512195123,10
7433,0.19951219512195123,10
7676,0.19951219512195123,10

ID
Element,Attribute,Weight
vertex0,attr,0.024146341463414635
vertex2,attr,0.01975609756097561
vertex33,attr,0.016829268292682928
vertex34,attr,0.02219512195121951
vertex4,attr,0.022682926829268292
vertex41,attr,0.014878048780487804
vertex7,attr,0.02170731707317073
vertex70,attr,0.018780487804878048
vertex76,attr,0.01780487804878049
vertex8,attr,0.02170731707317073
vertex80,attr,0.01780487804878049
vertex97,attr,0.016341463414634147

IDREF
Element,Attribute
\end{verbatim}
\end{scriptsize}
\end{multicols}

The first section deals with system information. Please note that some of these characteristics cannot be easily obtained programmatically and are thus stored in the source code as constants.\\
To obtain GLPK information, the program parses the first line of standard output produced by running \code{glpsol -v}. It tries to guess whether it's the Cygwin version by looking at the path to the binary.

The second section states the input file along with its size and graph representation (Section \ref{section-experiments-data}). Alpha and beta parameters for this instance belong here too.

\begin{footnotesize}
\begin{verbatim}
Configuration:
File name: graph.xml (101599 b)
  Graph representation: 82 vertices, 1101 edges
alpha: 1.0, beta: 1.0
\end{verbatim}
\end{footnotesize}

Results section opens stating the most important information first: how long did the experiment run and what was the highest and final quality (these two are potentially different). Numbers of attribute mappings in the best and final solution respectively are stated as well.

\begin{footnotesize}
\begin{verbatim}
Total time spent: 7754 ms
Final quality: 0.19951219512195123 (10 AMs)
Highest quality: 0.23463414634146343 (12 AMs)
\end{verbatim}
\end{footnotesize}

Construction phase results go next. Among reported information are the full identification of the heuristic (possibly along with its parameters), time taken, size of the pool created and the quality of the incumbent solution (again, with the number of its AMs).

\begin{footnotesize}
\begin{verbatim}
Algorithm: Random
  Time taken: 248 ms / Time since start: 248 ms
  Pool size: 10
  Quality: 0.19975609756097568 (11 AMs)
\end{verbatim}
\end{footnotesize}

Now for each of the improvement phases there is one section in output log. Information presented here has the same structure as with the construction phase. Please note that the \code{Pool size} is always measured \textit{after} the improvement run.

\begin{footnotesize}
\begin{verbatim}
Algorithm: Mutation, ratio = 0.1, limit = 1 s
  Time taken: 1512 ms / Time since start: 2710 ms
  Pool size: 11
  Quality: 0.21975609756097558 (11 AMs)
\end{verbatim}
\end{footnotesize}

After the last improvement phase, the reason why the metaheuristic terminated is stated. Possible causes are exceeding the maximum time available, maximum iterations or reaching the known optimum for this file and alpha / beta settings.
\\

To be able to reconstruct the progress of the metaheuristic, the next section contains CSV \nomenclature{CSV}{Comma Separated Values} formatted data for each iteration. Each row contains the time in milliseconds, quality of the incumbent solution and the number of its AMs.

\begin{footnotesize}
\begin{verbatim}
Time,Quality,AMs
...
841,0.15878048780487808,9
2710,0.21975609756097558,11
...
\end{verbatim}
\end{footnotesize}

And finally, it is important to know what is the ID/IDREF set recommended by this experiment run - the reason why we do all this! Thus the log is concluded by a CSV formatted list of element - attribute name pairs to be included in the ID and IDREF set, respectively.

\begin{footnotesize}
\begin{verbatim}
Element,Attribute,Weight
vertex0,attr,0.024146341463414635
...
\end{verbatim}
\end{footnotesize}

Note that in this example trace there were no IDREF AMs found. \qed