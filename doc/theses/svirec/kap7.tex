\chapter{Conclusion}

The aim of this thesis was a proposal and implementation of algorithm repairing XML functional dependencies violations. At first, existing solutions were analyzed and described. The result of the analyses was that none of the recent approaches uses, or in only limited form, user interaction in the process of finding suitable repair for the FD violation. Consequently, we proposed algorithm based on \cite{RepAndConsistentAnswer}, that uses user interaction in a various ways.\\

First, we incorporated a weight model into the XML data representation, which allows us to measure the modification cost of each possible repair candidate. This is also the first place for the user to interact in the repair process. Next, we have introduced a new concept of repair candidates clusters called repair groups. These clusters groups repair candidates according to the violation of which are repairing or part of the XML data that are modifying.

The main user interaction part of our algorithm is a selection of the repair group and consequently the repair candidate which is applied to the XML document. Besides the repair candidate selection, we also introduced a mechanism allowing to guess the next selection from the previous candidates selected by the user, if the user does not want to select the repair candidate anymore.

The experimental implementation of proposed algorithm and also the algorithm we based on was integrated into the jInfer framework. Both algorithms have been compared against each other on synthetic as well as real-world datasets. From the data gathered from the comparison is clear that our approach have found repairs with less modifications applied on the XML data. We also shown that with the user interaction is possible to change the usage of update operations used in repair, which may create more reasonable result for the user.

\section{Future Work}

Although our approach has introduced user in the process of repairing FD violations of the XML data, there is still work to be done. The first main task in the further work is to extend paths defining functional dependencies. With usage of more constructs that XPath provides, one can define more sophisticated FDs.

Another part of the algorithm that can be improved is clustering repair candidates into repair groups. The user can provide some additional criteria, that can change the resulting repair.

Last but not least, guessing part of the user selection algorithm can be improved by using more sophisticated heuristic algorithm to select proper repair candidate with regard to previous candidates selected by the user.
