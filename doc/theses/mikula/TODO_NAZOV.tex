\chapter{Diskusia k problemom}
\todo[inline]{Vhodny nazov pre tuto kapitolu}
Based on the previous chapter, we have quite wide space of possible utilization of XQuery queries. This chapter discusses some of questions and issues that emerged in an early phase of algorithm fabrication.

\section{Possibilities and Forms of Query Exploitation}
Queries can be exploited in several ways and levels. A basic query utilization can be achieved by analysis of queries without any other input data. A more complex method can utilize queries along with XML data. Another step can be evaluation of the queries using the XML data and consecutive analysis of the results. Even the process of evaluation can be analyzed to obtain some useful information. \todo[inline]{Uviest nejake priklady, co by sa dalo v ktorej faze zistovat?}

\todo[inline]{Dopisat, ako sme sa rozhodli. Este presne neviem, ale pravdepodobne prva faza bude spracovanie cisto queries a druha by mohla byt kontrola na zaklade XML dat. Na dalsie vstupne data asi nebude kapacita}

Another important question is how can be the queries processed. Will they be just searched for certain patterns like it is performed in method \cite{Necasky:2009:DXK:1529282.1529414} or will they be processed in a more sophisticated way? That could incorporate lexical and syntactic analyses \todo{referencie?} or even a form of an analysis of semantics \todo{referencia?}.

A result of the lexical and syntantic analyses can be some kind of so called syntax tree \todo{referencia}. It is a structure representing a word according to a formal grammar of some language \todo{definovat grammar, tree, language}. In our case, the language is XQuery, its grammar is defined in \todo{ref} and every query is a word of the XQuery language. Leafs of the tree represent terminals of the grammar while inner nodes represent non-terminals. \todo{priklad nejakeho stromu?} From point of view of this work, the syntax tree can be perceived as a preprocessed form of a query keeping its complete meaning and making its further processing more convenient. For instance, the tree can simplify a search for FLWOR statements. It is transitioned and nodes representing FLWORs are found. Then each subtree determined by one of the found nodes represents a FLWOR statement and it can be analyzed further.

The syntax tree can be also extended by additional information. An example is a static analysis of expression types. Types of literal expressions are defined, functions have return types, path expressions return nodes, etc. Applying rules defined in \todo{ref} can be determined types of complex expressions. These can be helpful for example in the analysis of built-in types of nodes.  
\todo[inline]{priklad}
\todo[inline]{Pre co sme sa rozhodli?}


\section{Inference of Structure}

1) odvodzovanie struktury z queries

2) pattern searching vs lexical, syntax and semantics analysis

3) vstupne data

4) napojenie na existujucu metodu

5) spravavanie dotazov - len ich analyza VS ich spustanie