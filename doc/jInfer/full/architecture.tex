\documentclass[a4paper,10pt,oneside]{article}
\usepackage{epsfig}
\usepackage{color}
\usepackage{url}
\usepackage[utf8]{inputenc}
\usepackage[T1]{fontenc}
\usepackage{amsmath}
\usepackage{amssymb}
\usepackage{import}

\newcommand{\myscale}{0.74}
\newcommand{\vect}[1]{\boldsymbol{#1}}
\newcommand{\code}[1]{\texttt{#1}}

\setlength{\hoffset}{-1in} %left margin will be 0, as hoffset is by default 1inch
\setlength{\voffset}{-1in} %analogous voffset
\setlength{\oddsidemargin}{1.5cm}
\setlength{\evensidemargin}{1.5cm}
\setlength{\topmargin}{1.5cm}
\setlength{\textheight}{26cm}
\setlength{\textwidth}{18cm}




\def\mftitle{jInfer Architecture}
\def\mfauthor{Michal Klempa, Mário Mikula, Robert Smetana, Michal Švirec, Matej Vitásek}
\def\mfadvisor{RNDr. Irena Mlýnková, Ph.D., Martin Nečaský, Ph.D.}
\def\mfplacedate{Praha, 2011}
\title{\bf\mftitle}
\author{\mfauthor \\ Advisors: \mfadvisor}
\date{\mfplacedate}

\ifx\pdfoutput\undefined\relax\else\pdfinfo{ /Title (\mftitle) /Author (\mfauthor) /Creator (PDFLaTeX) } \fi

\begin{document}
\maketitle
Target audience: developers willing to extend jInfer.\\

\textit{Note: we use the term \textbf{inference} for the act of creation of schema throughout this and other jInfer documents.}\\

The description of jInfer architecture will commence by describing the data
structures, namely representations of regular expressions and XML elements,
attributes and simple data.

Afterwards the interfaces of basic inference modules - Initial Grammar Gen-
erator, Simplifier and Schema Generator - will be explained.

Finally, the process of inference will be described.

\section{Package naming conventions}
All packages start with \code{cz.cuni.mff.ksi.jinfer}. Afterwards is the short, normalized name of the module (e.g. \code{base}) and finally the package structure in this module (e.g. \texttt{objects.utils}). All in all, a package in the Base module could look like 
\code{cz.cuni.mff.ksi.jinfer.base.objects.utils}

\section{Data structures}
\subsection{Regular expressions}

For general information on regular expressions, please refer to \cite{wikiregexp}, \cite{Hopcroft:2006:IAT:1177300}.

All classes pertaining to regular expressions can be found in the package \code{cz.cuni.mff.ksi.jinfer.base.regexp}.

In theory, regular expression is defined recursively:
\begin{itemize}
	\item Every symbol $x$ from alphabet is regular expression representing language ${x}$.
	\item If $R_1, R_2, \dots, R_n$ are regular expressions representing languages $L_1, L_2, \dots, L_n$, then $R_1 | R_2 | \dots | R_n$
	is regular expression representing language $\bigcup_{i = 1}^{n}{L_i}$. Operation $|$ is called alternation.
	\item If $R_1, R_2, \dots, R_n$ are regular expressions representing languages $L_1, L_2, \dots, L_n$, then $R_1, R_2 , \dots , R_n$
	is regular expression representing language $L_1 \cdot L_2 \cdot \ldots \cdot L_n$. Where $\cdot$ is languague concatenation,
	and operation $,$ is called concatenation as well.
	\item If $R$ is regular expression representing language $L$, then $R*$ 
	is regular expression representing language $L*$. In regular expressions, operator $*$ is called Kleene star.
\end{itemize}



In common practice, extended regular expressions are commonly used. Although they are not extending set of
languages representable by regular expressions, they give us nicer syntax. Extended regular expression adds syntax:
\begin{itemize}
	\item $R{m,n}$, where $m,n \in N$ with meaning of $R$ at least $m$-times, at most $n$-times,
	\item $R{m,}$ with meaning at least $m$-times,
	\item $R?$ with meaning $R{0,1}$,
	\item $R+$ with meaning $R{1,}$.
\end{itemize}

By this syntax, Kleene star is simply interval ${0,}$. This is just syntactic shortcut for long expresssions like $R, R, R, R?, R?$, which is just $R{3,5}$.

In jInfer, we implement regular expresssions as class \code{Regexp} with supplementing classes \code{RegexpInterval} and \code{RegexpType}.
Each \code{Regexp} instance has one of the \code{RegexpType} type:
\begin{itemize}
	\item Token - a letter of the language.
	\item Concatenation - one or more regular expression in an ordered sequence. Eg. $(a, b, c, d)$.
	\item Alternation - a choice between one or more regular expressions. Eg. $(a | b | c | d)$.
	\item Permutation - shortcut for all possible permutations of regular expressions (to work with \code{xs:all} in future).
	Our syntax to write down permutation is $(a  b c  d)$.
\end{itemize}

Each \code{Regexp} instance has interval associated with it (as a member). Class \code{RegexpInterval} represents all kinds of intervals
${m,n}; {m,}$. 

\nocite{*}
\bibliographystyle{alpha}
\bibliography{literature}

\end{document}
