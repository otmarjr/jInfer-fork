\documentclass[a4paper,10pt,oneside]{article}
\usepackage{graphicx}
\usepackage{color}
\usepackage{url}
\usepackage{subfigure}
\usepackage[utf8]{inputenc}
\usepackage[T1]{fontenc}
\usepackage{tgpagella}
%\usepackage[scale=0.9]{tgcursor}
%\usepackage[scale=0.9]{tgheros}
\usepackage{xstring}
\usepackage{wrapfig}

\newcommand{\myscale}{0.74}
\newcommand{\vect}[1]{\boldsymbol{#1}}
\newcommand{\code}[1]{\texttt{\StrSubstitute{#1}{.}{.\.}}}
\def\.{\discretionary{}{}{}}
\newcommand{\jmodule}[1]{\texttt{\textit{#1}}}

\setlength{\hoffset}{-1in} %left margin will be 0, as hoffset is by default 1inch
\setlength{\voffset}{-1in} %analogous voffset
\setlength{\oddsidemargin}{1.5cm}
\setlength{\evensidemargin}{1.5cm}
\setlength{\topmargin}{1.5cm}
\setlength{\textheight}{24cm}
\setlength{\textwidth}{18cm}

\def\mftitle{jInfer User Tutorial}
\def\mfauthor{Michal Klempa, Mário Mikula, Robert Smetana, Michal Švirec, Matej Vitásek}
\def\mfadvisor{RNDr. Irena Mlýnková, Ph.D., Martin Nečaský, Ph.D.}
\def\mfplacedate{Praha, 2011}
\title{\bf\mftitle}
\author{\mfauthor \\ Advisors: \mfadvisor}
\date{\mfplacedate}

\ifx\pdfoutput\undefined\relax\else\pdfinfo{ /Title (\mftitle) /Author (\mfauthor) /Creator (PDFLaTeX) } \fi

\begin{document}

\maketitle


 \section*{Tutorial}
        \par 
          Target audience: jInfer users. Anyone who needs to use jInfer to create DTD, XSD or another schema from existing XML documents, schemas (again: DTD, XSD, ...) or queries (such as XPath).
        
        \par 
          \emph{Note: we use the term \textbf{inference} for the act of creation of schema throughout this and other jInfer documents.}
        
\subsection*{Overview}
\begin{enumerate}
  \item Get \& install NetBeans.
  \item Get \& install jInfer.
  \item Infer schema.
\end{enumerate}
\subsection*{NetBeans}
\par 
  Do this if you don't have any NetBeans installed. If you have an older version than 6.9, jInfer will not run - please update it first.

\begin{enumerate}
  \item Go to \url{http://netbeans.org/}.
  \item Download NetBeans 6.9+, whichever bundle will do.
  \item Follow installation instructions.
  
\end{enumerate}
\subsection*{jInfer}
\par 
  Do this if you have the correct version of NetBeans installed. Note that jInfer features an automatic update utility - you don't need to follow these steps each time new version comes out.

\begin{enumerate}
  \item Go to \url{https://sourceforge.net/projects/jinfer/files/}.
  \item Download the latest jInfer binaries (jInfer-X.Y-bin.zip).
  \item Unpack the ZIP, you'll get a few NBM files.
  \item Start NetBeans, open \textit{Tools} > \textit{Plugins} from the main menu.
  \item Switch to \textit{Downloaded} tab.
  \item Click \textit{Add Plugins...}.
  \item Select all the unpacked NBM files.
  \item Click \textit{Install} and follow the instructions.
  \item Restart NetBeans.
  \end{enumerate}
\subsection*{Basic Inference}
\par 
  This section assumes you have NetBeans and jInfer installed. After the first run, a jInfer Welcome window will appear, guiding you through basically the same steps that will be covered here. You might want to follow them instead.

\par 
  Let's assume that you have one or more XML files, and you want to create XSD schema from them.

\begin{enumerate}\item 
 Create a new jInfer project.
    \begin{enumerate}\item 
        \textit{File} > \textit{New Project} \textbf{or} \textit{New Project} icon from the toolbar \textbf{or} right-click the area in \textit{Projects} window > \textit{New Project}.
      \item Select \textit{jInfer} category, \textit{jInfer Project}.
      \item Click \textit{Next}.
      \item Choose a name (for example MyInference) and location of your project.
      \item Click \textit{Next}.
      \item In the \textit{Schema Generator} combo box select \textit{Basic XSD Exporter}.
      \item Click \textit{Finish}.
      \end{enumerate}
  \item Add input XML files to this project.
    \begin{enumerate}\item 
  Right click the newly created project in \textit{Projects} window.
      \item Select \textit{Add files}.
      \item Select XML files you want to use for schema inference.
      \item If you now open the \textit{XML} folder in your project, you will see the added files.
      \end{enumerate}
  \item Run inference.
    \begin{enumerate}\item 
        \textit{Run Project} in the toolbar (green play icon) \textbf{or} right click the project, select \textit{Run}.
      \item Generated schema will open in a new \textit{Editor} window. It will also appear in the \textit{Output} folder.
      \item You will find the file with the schema in your project (location from step 1d, folder \textit{output}).
      \end{enumerate}
  \end{enumerate}
\subsection*{Project files}\par 
  This section assumes you have successfully inferred at least one schema.

\begin{itemize}\item 
 To add input files to a specific folder (XML documents, schemas, queries) without relying on jInfer to guess the folder from the extension, right click the folder and select \textit{Add XYZ files}.
  \item Note that input folders are virtual - files in them still reside at their original locations. You can't actually delete them from disk from within jInfer, just remove them from these virtual folders by selecting \textit{Delete} from their context menu.
  \item \textit{Output} folder is, on the other hand, a faithful representation of the folder \textit{output} inside your jInfer project (step 1d). If you delete a schema here, it will be deleted from the disk too.
  \item You might want to see a diff between two input documents, or old and inferred schema - just select both files at once, right-click and select \textit{Tools} > \textit{Diff} from their context menu.
  \end{itemize}
\subsection*{Setting up jInfer}
\par 
  This section assumes that you have successfully inferred at least one schema and know how to manipulate input and output files in a project.

\par 
  To change jInfer settings, you have to first realize whether it is a change affecting a single project at a time (for example, changing the output language from XSD to DTD) or whole jInfer at once (whether schema should open in Editor window after the inference finishes).

\begin{enumerate}\item 
 First type of settings is found in so called Project Preferences. You can access them from the respective jInfer project's context menu: right click the project and select \textit{Properties}.
  \item Second type is called jInfer Options and can be found among NetBeans' options: open them (Windows, Linux: \textit{Tools} > \textit{Preferences...}, Mac: \textit{NetBeans} > \textit{Preferences...}) and switch to \textit{jInfer} tab.
  \end{enumerate}
\subsection*{Interactive simplification}
\par 
  This section assumes that you have successfully
inferred a few schemas and would like to play with the process, interfere with it.

\par 
  Default simplifier that jInfer uses is non-interactive: after it is run, it will
work for a while and then display the generated schema. However, jInfer bundles also
a way to guide the inference process interactively, by selecting states in an
automaton that should be merged together and selecting which state should be
removed while converting an automaton to a regular expression. For more
information on this, see the documentation of TwoStep simplifier.
To try out interactive inference, do the following with your jInfer project:

\begin{enumerate}
  \item Open \textit{Project Preferences}.
  \item Navigate to \textit{Simplifiers} > \textit{TwoStep} >
      \textit{Automaton Merging State}.
  \item In the \textit{Automaton simplifier} combobox select
      \textit{User Interactive}.
  \item Navigate to \textit{State Removal}.
  \item In the \textit{State Removal orderer} combobox select
      \textit{User Interactive}.
  \item Save the preferences by clicking \textit{OK}.
  \item Run the inference again.
  \item A new window will appear, which will require you
      to select states to be merged in each step of simplification.
      After selecting these states, proceed by clicking \textit{Continue}.
\end{enumerate}


 

\end{document}
