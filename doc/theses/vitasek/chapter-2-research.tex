\chapter{Research}
\label{chapter-research}

According to the article \cite[Chapter~4]{fidax}, the problem of finding an ID set with weight more than some $K$ ($K$-\textsc{IDSet}) is in NP. Furthermore, the independent set (IS) problem can be reduced to $K$-\textsc{IDSet}, meaning $K$-\textsc{IDSet} is NP-hard and thus NP-complete.

The transformation from IS problem formulation to $K$-\textsc{IDSet} problem formulation is as follows.

\begin{quote}
Let $G = (V, E)$ be a simple connected graph with vertex set $V = \{v_1, \ldots, v_n\}$, and edge set $E = \{e_1, \ldots, e_m\}$. We define the attribute mappings as follows. Let $ \mathcal{I} = V \cup E$, and define $value(x) = x, x \in \mathcal{I}$. For each vertex $v_i \in V$, we create a mapping $m_i = \{(v_i, e_j): e_j \in E \,\text{is incident on}\, v_i \}$, and define $\tau(m_i) = v_i$; let $C = \{m_1, \ldots, m_n\}$ be set of all such mappings. It is clear that G has an independent set of size $K$ iff $C$ has an ID set of size $K$. Also, $C$ can be constructed in time polynomial on $n+m$.
\end{quote}

The article continues by proving that the problem of finding maximum weighted IS can be reduced to the problem of finding an ID set with maximum weight (\textsc{Max-IDSet}). This again means that \textsc{Max-IDSet} is NP-complete and, furthermore, unless $\P = \NP$, \textsc{Max-IDSet} has no constant factor approximation algorithm.

The difference in transformation from maximum weighted IS to \textsc{Max-IDSet} is as follows.

\begin{quote}
[...] with the added restriction that $w(m_i) = w(v_i), v_i \in V$.
\end{quote}

Note that the transformation works in both ways: it is equivalently possible to create a maximum weighted IS instance for a given \textsc{Max-IDSet} instance.

The article further suggests a heuristic approach described in Section \ref{section-mip-fidax}, which was incorporated into the framework proposed by this work.

To the best of our knowledge, there are no other articles dealing directly with this problem.\\

Fajt in \cite{fajt} introduces several algorithms to find XML keys (see \cite{keX}) in existing XML documents. Simple keys found this way can be translated to an equivalent \texttt{ID} attribute definition. The process is described in \cite[Ch.\ 9, s.\ 3]{vlist2002xml}. This opens a new line of research: finding XML keys using an algorithm modified to look only for \textit{useful} keys and then converting them to \texttt{ID} attributes.\\

Maximum weigthed IS is a well researched topic with a lot of known direct or approximation algorithms, see e.g. \cite{JM1986425} or \cite{Fomin:2009:MCA:1552285.1552286}.\\