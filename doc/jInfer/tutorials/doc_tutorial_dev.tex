\documentclass[a4paper,10pt,oneside]{article}
\usepackage{graphicx}
\usepackage{color}
\usepackage{url}
\usepackage{subfigure}
\usepackage[utf8]{inputenc}
\usepackage[T1]{fontenc}
\usepackage{tgpagella}
%\usepackage[scale=0.9]{tgcursor}
%\usepackage[scale=0.9]{tgheros}
\usepackage{xstring}
\usepackage{wrapfig}

\newcommand{\myscale}{0.74}
\newcommand{\vect}[1]{\boldsymbol{#1}}
\newcommand{\code}[1]{\texttt{\StrSubstitute{#1}{.}{.\.}}}
\def\.{\discretionary{}{}{}}
\newcommand{\jmodule}[1]{\texttt{\textit{#1}}}

\setlength{\hoffset}{-1in} %left margin will be 0, as hoffset is by default 1inch
\setlength{\voffset}{-1in} %analogous voffset
\setlength{\oddsidemargin}{1.5cm}
\setlength{\evensidemargin}{1.5cm}
\setlength{\topmargin}{1.5cm}
\setlength{\textheight}{24cm}
\setlength{\textwidth}{18cm}

\def\mftitle{jInfer Module Developer's Tutorial}
\def\mfauthor{Michal Klempa, Mário Mikula, Robert Smetana, Michal Švirec, Matej Vitásek}
\def\mfadvisor{RNDr. Irena Mlýnková, Ph.D., Martin Nečaský, Ph.D.}
\def\mfplacedate{Praha, 2011}
\title{\bf\mftitle}
\author{\mfauthor \\ Advisors: \mfadvisor}
\date{\mfplacedate}

\ifx\pdfoutput\undefined\relax\else\pdfinfo{ /Title (\mftitle) /Author (\mfauthor) /Creator (PDFLaTeX) } \fi

\begin{document}

\maketitle

\section*{Module Developer's Tutorial}
\par 
  Target audience: jInfer module developers. Anyone who needs to extend jInfer capabilities by writing a new module.

\par 
  
    This tutorial assumes that you are a seasoned Java developer. Having
    experience with programming in some kind of framework (NetBeans Platform
    above all) will help you a lot.
    Make sure you have read the article on
    architecture, data structures and inference process to understand what you
    will be implementing. Having read the documentation for
    remaining modules will help you too.
    Also, before starting this tutorial, make sure you can
    build jInfer from sources.
  

\subsection*{Overview}
  \begin{enumerate}
    \item Get NetBeans, jInfer sources, try a build.
    \item Decide on the type of module you want to create.
    \item Create new NetBeans module.
    \item Implement jInfer-specific interfaces.
    \item Implement your logic.
  \end{enumerate}
\subsection*{Building jInfer from sources}
  \par 
    Refer to the official instructions.
  
\subsection*{What type of module?}
  \par 
    First thing you need to realize is what kind of module you will be
    implementing. Is it going to be a part of the inference? If yes, what stage?
    Importing the initial grammar? Simplifying it? Exporting to resulting schema?
    If you are not sure what these terms mean, go read the articles on interence process again.
  
  \par 
    If your logic doesn't belong to the inference process, it might just extend
    one of the existing modules. Try looking for the code you would like to change.
  
\subsection*{New NetBeans module}
\par 
  This section assumes you have successfully imported jInfer in NetBeans.

\begin{enumerate}
  \item Expand the jInfer suite (brown puzzles icon).
  \item Right-click \textit{Modules}, select \textit{Add New...}.
  \item On the first page, pick a name for your module. Click \textit{Next}.
  \item On the second page, fill out the following:
    \begin{itemize}
      \item 
        \textit{Code base}: this will be the package structure in which your code is placed.
        If you want, follow our convention: \texttt{cz.cuni.mff.ksi.jinfer.\textit{yourmodule}}.
        Otherwise choose something like \texttt{\textit{tld.company.jinfer.yourmodule}}.
      
      \item \textit{Module display name}: pretty obvious.
      \item \textit{Generate XML layer}: you may want to check this option, but it doesn't really matter at this stage.
    \end{itemize}
    Rest of the options is uninteresting in most cases.
  
  \item Click \textit{Finish}.
\end{enumerate}
\subsection*{Implement jInfer's specifics}
\par 
This section will show how to deal with a new inference module - a simplifier
that does no actual simplification, just returns the grammar it got on input.
Your module will need a "main" class implementing the chosen inference
interface, annotated with a \texttt{@ServiceProvider} annotation.

\par 
But first, we need to do some setup.
\begin{enumerate}
  \item Right-click the newly created module, select \textit{Properties}.
  \item In \textit{Libraries > Module Dependencies}, click \textit{Add Dependency...}.
  \item Filter for "base" and select \textit{Base} as a dependency. Click \textit{OK}.
  \item 
      Still in \textit{Properties}, switch to \textit{Display} category. Fill in:
      \begin{itemize}
        \item \textit{Display Name}: Fill in a user-friendly name of your module.
        \item \textit{Display Category}: type in \textit{jInfer}.
        \item \textit{Short \& Long Description}: unleash your inner poet!
      \end{itemize}
  
  \item Still in \textit{Properties}, switch to \textit{API Versioning}.
  \item Type \texttt{cz.cuni.mff.ksi.jinfer.base.interfaces.inference.Simplifier} in \textit{Provided Tokens}.
  \item Close \textit{Properties} by clicking \textit{OK}.
\end{enumerate}

\par 
Now to the class itself.
\begin{enumerate}
  \item Add a new Java class: right click \textit{Source Packages} in your new
      module, select \textit{New > Java Class...}.
  \item Fill in class name, for example \textit{MySimplifierImpl}.
  \item Fill in package based on the \textit{Code base} you selected while creating
      the module itself.
  \item Click \textit{Finish}.
  \item In the heading line of the class, add \texttt{implements Simplifier} and
      fix the imports (you need \texttt{Simplifier} from Base module).
  \item Annotate this class with \texttt{@ServiceProvider(service = Simplifier.class)}.
  \item NetBeans will complain about missing method implementations, add them.
  \item 
    Now fill in method bodies:
    \begin{itemize}
      \item \texttt{getName()}: return a string with module unix name, for example \texttt{"mysimplifier"}.
      \item \texttt{getDisplayName()}: return a user friendly module name, for example \texttt{"My First Simplifier"}.
      \item \texttt{getModuleDescription()}: return a short module description. You can also return \texttt{getDisplayName()} in this case.
      \item \texttt{getCapabilities()}: for the moment, it is enough to return \texttt{Collections.emptyList()}.
      \item \texttt{start()}: here is where your main logic belongs. In the case of a simplifier, you would do something with the rules and return a simplified grammar. For now, just do \texttt{callback.finished(initialGrammar);}
    \end{itemize}
  
\end{enumerate}

\subsection*{Run jInfer}
\par 
  At this moment, run the whole jInfer suite from the NetBeans you imported it into.
  A new, child NetBeans should open with your module correctly installed. You can
  now follow the User tutorial, just select your new
  simplifier while creating a new jInfer project.
  That's it! Start hacking you logic now!

\subsection*{Implement your logic}
\par 
  This is the part where you actually have to do some thinking :-) Implementing
  an importer? Take that \texttt{InputStream} and create some rules of it! Simplifier?
  Take the rules you got and compact them somehow! Exporter? Take those rules and
  write them out as a \texttt{String}! It only depends on which module you are
  in and algorithm you're implementing.

\subsection*{Dealing with jInfer and NetBeans Platform}
\par 
  You will soon get into situation where you need to interact either with
  jInfer or directly NetBeans. There is a
  tutorial for dealing with these cases.



\end{document}
