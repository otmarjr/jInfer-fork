\documentclass[a4paper,10pt,oneside]{article}
\usepackage{graphicx}
\usepackage{color}
\usepackage{url}
\usepackage{subfigure}
\usepackage[utf8]{inputenc}
\usepackage[T1]{fontenc}
\usepackage{tgpagella}
%\usepackage[scale=0.9]{tgcursor}
%\usepackage[scale=0.9]{tgheros}
\usepackage{xstring}

\newcommand{\myscale}{0.74}
\newcommand{\vect}[1]{\boldsymbol{#1}}
\newcommand{\code}[1]{\texttt{\StrSubstitute{#1}{.}{.\.}}}
\def\.{\discretionary{}{}{}}
\newcommand{\jmodule}[1]{\texttt{\textit{#1}}}

\setlength{\hoffset}{-1in} %left margin will be 0, as hoffset is by default 1inch
\setlength{\voffset}{-1in} %analogous voffset
\setlength{\oddsidemargin}{1.5cm}
\setlength{\evensidemargin}{1.5cm}
\setlength{\topmargin}{1.5cm}
\setlength{\textheight}{24cm}
\setlength{\textwidth}{18cm}

\def\mftitle{jInfer ProjectType Module Description}
\def\mfauthor{Michal Klempa, Mário Mikula, Robert Smetana, Michal Švirec, Matej Vitásek}
\def\mfadvisor{RNDr. Irena Mlýnková, Ph.D., Martin Nečaský, Ph.D.}
\def\mfplacedate{Praha, 2011}
\title{\bf\mftitle}
\author{\mfauthor \\ Advisors: \mfadvisor}
\date{\mfplacedate}

\ifx\pdfoutput\undefined\relax\else\pdfinfo{ /Title (\mftitle) /Author (\mfauthor) /Creator (PDFLaTeX) } \fi

\begin{document}
\maketitle
\noindent Target audience: developers willing to extend jInfer, specifically extend jInfer project structure.

\noindent \begin{tabular}{|l|l|} \hline
Responsible developer: & Michal Švirec \\ \hline
Required tokens:       & cz.cuni.mff.ksi.jinfer.base.interfaces.inference.IGGenerator \\
 & cz.cuni.mff.ksi.jinfer.base.interfaces.inference.SchemaGenerator \\
 & cz.cuni.mff.ksi.jinfer.base.interfaces.inference.Simplifier \\
 & org.openide.windows.IOProvider \\ \hline
Provided tokens:       & none \\ \hline
Module dependencies:   & Base \\
	& Runner \\ \hline
Public packages:       & cz.cuni.mff.ksi.jinfer.projecttype.actions \\ \hline
\end{tabular}

\section{Introduction}

\jmodule{ProjectType} is the module responsible for creation of NBP project type which groups input/output files that belongs to one specific inference. Each jInfer project also allows user to set properties specific for inference.

\section{Structure}

Structure of \jmodule{ProjectType} can be divided into following five main parts.
\begin{itemize}
	\item Base classes - Classes providing main functionality like creation of project, defining operations like move, delete, copy etc. All the base classes are contained in the \code{cz.cuni.mff.ksi.jinfer.projecttype} package.
	\item Visualization classes - These classes create tree structure of project in NBP Projects view and are contained in the \code{cz.cuni.mff.ksi.jinfer.projecttype.nodes}
	\item Actions - Classes from \code{cz.cuni.mff.ksi.jinfer.projecttype.actions} package that provides actions allowing adding, removing input files into project, or running the project.
	\item Properties - Classes responsible for creation of project properties window with properties category tree. These are situated in  \code{cz.cuni.mff.ksi.jinfer.projecttype.properties} package.
	\item Project wizard - Classes representing creation of project through new project wizard from  \code{cz.cuni.mff.ksi.jinfer.projecttype.sample} package.
\end{itemize}

\subsection{Base classes}



\subsection{Visualization classes}

\subsection{Actions}

\subsection{Properties}

\subsection{Project wizard}

\subsection{Preferences}


\nocite{*}
\newpage
\bibliographystyle{alpha}
\bibliography{literature}

\end{document}
