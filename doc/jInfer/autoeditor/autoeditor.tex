\documentclass[a4paper,10pt,oneside]{article}
\usepackage{graphicx}
\usepackage{color}
\usepackage{url}
\usepackage{subfigure}
\usepackage[utf8]{inputenc}
\usepackage[T1]{fontenc}
\usepackage{tgpagella}
%\usepackage[scale=0.9]{tgcursor}
%\usepackage[scale=0.9]{tgheros}
\usepackage{xstring}

\newcommand{\myscale}{0.74}
\newcommand{\vect}[1]{\boldsymbol{#1}}
\newcommand{\code}[1]{\texttt{\StrSubstitute{#1}{.}{.\.}}}
\def\.{\discretionary{}{}{}}
\newcommand{\jmodule}[1]{\texttt{\textit{#1}}}

\setlength{\hoffset}{-1in} %left margin will be 0, as hoffset is by default 1inch
\setlength{\voffset}{-1in} %analogous voffset
\setlength{\oddsidemargin}{1.5cm}
\setlength{\evensidemargin}{1.5cm}
\setlength{\topmargin}{1.5cm}
\setlength{\textheight}{24cm}
\setlength{\textwidth}{18cm}

\def\mftitle{jInfer AutoEditor Module Description}
\def\mfauthor{Michal Klempa, Mário Mikula, Robert Smetana, Michal Švirec, Matej Vitásek}
\def\mfadvisor{RNDr. Irena Mlýnková, Ph.D., Martin Nečaský, Ph.D.}
\def\mfplacedate{Praha, 2011}
\title{\bf\mftitle}
\author{\mfauthor \\ Advisors: \mfadvisor}
\date{\mfplacedate}

\ifx\pdfoutput\undefined\relax\else\pdfinfo{ /Title (\mftitle) /Author (\mfauthor) /Creator (PDFLaTeX) } \fi

\begin{document}
\maketitle
\noindent Target audience: developers willing to extend jInfer, specifically alter displaying of automata .

\noindent \begin{tabular}{|l|l|} \hline
Responsible developer: & Mário Mikula \\ \hline
Required tokens:       & org.openide.windows.WindowManager \\ \hline
Provided tokens:       & none \\ \hline
Module dependencies:   & Base \\ 
					   & JUNG \\ \hline
Public packages:       & cz.cuni.mff.ksi.jinfer.autoeditor \\ 
					   & cz.cuni.mff.ksi.jinfer.autoeditor.automatonvisualizer \\
   					   & cz.cuni.mff.ksi.jinfer.autoeditor.automatonvisualizer.layouts \\
   					   & cz.cuni.mff.ksi.jinfer.autoeditor.gui.component \\ \hline
\end{tabular}

\section{Introduction}

This is an implementation of a \jmodule{AutoEditor}. Using JUNG library, it provides an API to display and user interactively modify automata, so the process of inference can be easily made user interactive.

\section{Structure}

Structure of \jmodule{AutoEditor} can be divided into following 3 main parts.

\begin{itemize}
	\item API - API to display automaton in GUI.
	\item Base classes - Classes providing basic functionality that can be extended and combined to achieve desired visualization of an automaton.
	\item Derived classes - Classes derived from the base classes that are used in existing modules and simultaneously serve as examples.
\end{itemize}

First, base classes and a creation of automaton visualization will be described.

\subsection{Base classes}

Main two classes representing visualization of automaton are \code{Visualizer} and \code{AbstractComponent}. \code{Visualizer} is a graphical representation of automaton and \code{AbstractComponent} is a panel (extends \code{JPanel}) containing the \code{Visualizer} which will be displayed in GUI.
TODO obrazok ako AC dedi od JPanelu a obsahuje Visualizer.

\subsubsection{Visualizer}

TODO translate

Trieda \code{Visualizer} dedi od JUNGoveho \code{VisualizationViewer}, takze poskytuje vsetky jeho metody a navyse podporu pre ulozenie automatu do obrazku - metody \code{saveImage()} a \code{getSupportedImageFormatNames()}. Pre ulozenie obrazku vsak tieto metody nie je nutne volat, pretoze AutoEditor GUI obsahuje tlacitko na ulozenie prave vykresleneho automatu do obrazku (viz dalej).

Constructor method bere ako argument instanciu triedy \code{Layout}. Viac informacii o tejto triede a jej pouzitie v kapitole TODO ref.

TODO obrazok ako Visualizer dedi od VisualizationVieweru a obsahuje Layout.

\subsubsection{AbstractComponent}

TODO translate

Trieda \code{AbstractComponent} je panel v ktorom bude vykresleny automat, presnejsie Visualizer reprezentujuci nejaky automat. Dedi od triedy \code{JPanel}, takze poskytuje vsetky jej metody a spravanie. Navyse poskytuje metody

setVisualizer()
getVisualizer()
waitForGuiDone()
guiDone()
guiInterrupt()
guiInterrupted()

a abstract metodu
getAutomatonDrawPanel()

Purpose tejto triedy je rozsirit ju a poskladat si panel aky sa hodi (tlacitka, napisy, ...) s tym, ze musi obsahovat aspon jeden JPanel, v ktorom bude vykresleny nastaveny Visualizer. Ucel metody getAutomatonDrawPanel() je vratit tento JPanel, aby AutoEditor vedel, kam ma ten Visualizer vykreslit.

Ak je ziadany user interaktivita, je nutne si podporu pre nu zahrnut prave do tejto triedy. Pre viac informacii viz TODO ref.

Visualizer sa nenastavuje v konstruktore z toho dovodu, ze casto je ziaduce, aby sa na rovnakom paneli kreslilo postupne viac roznych automatov. Na to nie je nutne vyrabat novu instanciu, ale staci na jednej instancii volat setVisualizer().

\subsection{API}

API AE je velmi jednoduche. Trieda \code{AutoEditor} poskytuje tieto 3 staticke metody.

drawComponentAsync()
drawComponentAndWaitForGUI()
closeTab()

\subsection{Derived classes}

StatePickingVisualizer
StatesPickingVisualizer

\subsection{GUI}

TODO

tlacitka

\subsection{Preferences}

TODO

All settings provided by \jmodule{BasicXSDExporter} are project-wide, the preferences panel is in \code{cz.cuni.mff.ksi.jinfer.basicxsd.properties} package. As mentioned above, it is possible to set the following. 

\begin{itemize}
	\item Turn off generation of global element types. Turning off this feature is not recommended as it may cause certain problems with validity of resulting XSD. See \ref{section:problems}.
	\item Minimal number of occurrences of element to define its type globally. (Only if generation of global elements is active.)
	\item Number of spaces in output per one level of indentation.
	\item Global type name prefix. It is a string which will be inserted before a name of a type, which is derived from element's name. Can be also an empty string. (Only if generation of global elements is active.)
	\item Global type name suffix. It is a string which will be appended after a name of a type, which is derived from element's name. Can be also an empty string. (Only if generation of global elements is active.)
\end{itemize}


\nocite{*}
\newpage
\bibliographystyle{alpha}
\bibliography{literature}

\end{document}
