\documentclass[a4paper,10pt,oneside]{article}
\usepackage{graphicx}
\usepackage{color}
\usepackage{url}
\usepackage{subfigure}
\usepackage[utf8]{inputenc}
\usepackage[T1]{fontenc}
\usepackage{tgpagella}
%\usepackage[scale=0.9]{tgcursor}
%\usepackage[scale=0.9]{tgheros}
\usepackage{xstring}

\newcommand{\myscale}{0.74}
\newcommand{\vect}[1]{\boldsymbol{#1}}
\newcommand{\code}[1]{\texttt{\StrSubstitute{#1}{.}{.\.}}}
\def\.{\discretionary{}{}{}}
\newcommand{\jmodule}[1]{\texttt{\textit{#1}}}

\setlength{\hoffset}{-1in} %left margin will be 0, as hoffset is by default 1inch
\setlength{\voffset}{-1in} %analogous voffset
\setlength{\oddsidemargin}{1.5cm}
\setlength{\evensidemargin}{1.5cm}
\setlength{\topmargin}{1.5cm}
\setlength{\textheight}{24cm}
\setlength{\textwidth}{18cm}

\def\mftitle{jInfer BasicDTDExporter Module Description}
\def\mfauthor{Michal Klempa, Mário Mikula, Robert Smetana, Michal Švirec, Matej Vitásek}
\def\mfadvisor{RNDr. Irena Mlýnková, Ph.D., Martin Nečaský, Ph.D.}
\def\mfplacedate{Praha, 2011}
\title{\bf\mftitle}
\author{\mfauthor \\ Advisors: \mfadvisor}
\date{\mfplacedate}

\ifx\pdfoutput\undefined\relax\else\pdfinfo{ /Title (\mftitle) /Author (\mfauthor) /Creator (PDFLaTeX) } \fi

\begin{document}
\maketitle
\noindent Target audience: developers willing to extend jInfer, specifically hack the DTD export.

\noindent \begin{tabular}{|l|l|} \hline
Responsible developer & Matej Vitásek \\ \hline
Required tokens       & none \\ \hline
Provided tokens       & \code{cz.cuni.mff.ksi.jinfer.base.interfaces.inference.SchemaGenerator} \\ \hline
Module dependencies   & Base \\ \hline
Public packages       & none \\ \hline
\end{tabular}

\section{Introduction}

This is a relatively simple implementation of a \jmodule{SchemaGenerator} exporting the inferred schema to DTD.

\section{Structure}

The main class implementing \code{SchemaGenerator} interface and simultaneously being registered as its service provi\-der is \code{Sche\.ma\.Ge\.ne\.ra\.tor\.Impl}. Its \code{start()} method first topologically sorts all rules (elements) it got in the simplified grammar on input. This sorting is necessary to avoid using anything not yet defined in the resulting schema. Afterwards, it creates their DTD string represenation.\\

Export of a single element is handled in the \code{elementToString()} method. First the actual \code{<!ELEMENT \ldots>} tag is exported, and after that, its attributes in \code{<!ATTLIST \ldots>} tag (if there are any).

\subsection{Element content export}
Elements are processed by \code{elementToString()} method.
Whole regexp of an element is sent to method \code{expandIn\.ter\.vals\.Reg\.exp()} of class \code{IntervalExpander}.
Its purpose is to convert intervals of a regexp and its children to those representable in DTD.
For example regular expression $(a\{2,5\}, b\{0,2\})$ would be transformed to $(a,a,a?,a?,a?,b?,b?)$.

Class \code{IntervalExpander} works recursively.
Regexp is passed to method \code{expandIntervalsRegexp()} to handle regexp in that element.
This method contains a big\code{switch} based on the regexp type.
For $\lambda$, it returns $\lambda$.
Otherwise, it examines interval of regexp in method \code{is\.Sa\.fe\.In\.ter\.val}.
In DTD one can represent $+, ?, \ast$, so safe intervals are $\{1, \infty\}, \{0, 1\}, \{0, \infty\}$ respectively.
If an interval is not safe, it has to be expanded.
It is easy to do so, first output $min$-times the regexp itself - that is the minimum occurences, with interval set to $\{1, 1\}$.
Then, if interval is bounded, output $max - min$-times the regexp itself with interval $\{0, 1\}$ - that is optional part.
If it is unbounded and $min$ is zero, attach the regexp once, with interval $\{0, \infty\}$.
If $min$ is non-zero, last regexp in required string gets interval $\{1, \infty\}$.

After intervals are expanded, resulting regexp is passed to \code{regexpToString()}, which contains a big \code{switch} statement based on the type of regexp.
For $\lambda$, it simply returns \code{EMPY} as string.
Tokens are at first examined if they are \code{SimpleData}; if so, string \code{\#PCDATA} is returned.
If not, element name is returned.
If interval of this regexp is different from $\{1, 1\}$, the interval \code{toString()} representation is appended.\\

Situation is a bit complicated with complex regexps that contain \code{SimpleData} somewhere inside the tree.
They are processed in \code{comboToString} method which first checks if there are no simple data in whole tree.
If not, regexp can be outputted just as list, e.g. $(a, b, c)$ or $(a | b | c)$ or $(a \& b \& c)$.
If there is at least one simple data, flattening is applied.
That means, all elements from regexp are collected into one flat list.
All simple datas are trashed away.
On output is string \code{(\#PCDATA, a, b, c, d)*}, as this is the only way to represent mixed content in DTDs.

\subsection{Attribute export}

Code exporting attributes is in \code{attributeToString()}. First thing this method does is to assess the domain of a particular atribute: this is a map indexed by attribute values containing number of occurences for each such attribute. Type definition of an attribute is generated in the \code{DomainUtils.getAttributeType()} method. Based on a user setting, this might decide to enumerate all possible values of this attribute using the \code{(a|b|c)} notation, otherwise it just returns \code{\#CDATA}.\\
Attribute requiredness is assessed based on \code{required} metadata presence. If an attribute is not deemed required, it might have a default value: if a certain value is prominent in the attribute domain (based on user setting again), it is declared default.

\subsection{Preferences}

All settings provided by \jmodule{BasicDTDExporter} are project-wide, the preferences panel is in \code{cz.cuni.mff.ksi.jinfer.basicdtd.properties} package. As mentioned before, it is possible to set the following. 
\begin{itemize}
	\item Maximum attribute domain size which is exported as a list of all values (\code{(a|b|c)} notation).
	\item Minimal ratio an attribute value in the domain needs to have in order to be declared default.
\end{itemize}

\section{Data flow}

Flow of data in this module is following.
\begin{enumerate}
	\item \code{SchemaGeneratorImpl} topologically sorts elements (rules) it got on input.
	\item For each element, relevant portion of DTD schema is generated.
	\item String representation of the schema is returned along with the information that file extension should be ``dtd''.
\end{enumerate}

\nocite{*}
\newpage
\bibliographystyle{alpha}
\bibliography{literature}

\end{document}
