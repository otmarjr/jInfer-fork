\documentclass[a4paper,10pt,oneside]{article}
\usepackage{graphicx}
\usepackage{color}
\usepackage{url}
\usepackage{subfigure}
\usepackage[utf8]{inputenc}
\usepackage[T1]{fontenc}
\usepackage{tgpagella}
%\usepackage[scale=0.9]{tgcursor}
%\usepackage[scale=0.9]{tgheros}

\newcommand{\myscale}{0.74}
\newcommand{\vect}[1]{\boldsymbol{#1}}
\newcommand{\code}[1]{\texttt{#1}}
\newcommand{\jmodule}[1]{\texttt{\textit{#1}}}

\setlength{\hoffset}{-1in} %left margin will be 0, as hoffset is by default 1inch
\setlength{\voffset}{-1in} %analogous voffset
\setlength{\oddsidemargin}{1.5cm}
\setlength{\evensidemargin}{1.5cm}
\setlength{\topmargin}{1.5cm}
\setlength{\textheight}{24cm}
\setlength{\textwidth}{18cm}

\def\mftitle{jInfer Base Module Description}
\def\mfauthor{Michal Klempa, Mário Mikula, Robert Smetana, Michal Švirec, Matej Vitásek}
\def\mfadvisor{RNDr. Irena Mlýnková, Ph.D., Martin Nečaský, Ph.D.}
\def\mfplacedate{Praha, 2011}
\title{\bf\mftitle}
\author{\mfauthor \\ Advisors: \mfadvisor}
\date{\mfplacedate}

\ifx\pdfoutput\undefined\relax\else\pdfinfo{ /Title (\mftitle) /Author (\mfauthor) /Creator (PDFLaTeX) } \fi

\begin{document}
\maketitle
\noindent Target audience: developers willing to extend jInfer.

\noindent \begin{tabular}{|l|l|} \hline
Responsible developer: & Matej Vitásek \\ \hline
Required tokens:       & none \\ \hline
Provided tokens:       & none \\ \hline
Module dependencies:   & none \\ \hline
Public packages:       & cz.cuni.mff.ksi.jinfer.base.automaton \\ 
& cz.cuni.mff.ksi.jinfer.base.interfaces \\ 
& cz.cuni.mff.ksi.jinfer.base.interfaces.inference \\ 
& cz.cuni.mff.ksi.jinfer.base.interfaces.nodes \\
& cz.cuni.mff.ksi.jinfer.base.objects \\ 
& cz.cuni.mff.ksi.jinfer.base.objects.nodes \\ 
& cz.cuni.mff.ksi.jinfer.base.regexp \\ 
& cz.cuni.mff.ksi.jinfer.base.utils \\
& org.apache.log4j.* \\ \hline
\end{tabular}

\section{Introduction}

This is the module containing data structures, interfaces and logic shared across the whole jInfer framework. Virtually every other module containing logic should in theory depend on \jmodule{Base}.

\section{Structure}

Description of \jmodule{Base} structure will partially mirror its JavaDoc documentation. For more detailed information, refer to it directly.\\

\jmodule{Base} contains logic for Log4j initialization in \code{Installer} class. Configuration of the overall logging granularity level (NetBeans options integration) is contained in \code{cz.cuni.mff.ksi.jinfer.base.options} package. Shared jInfer graphics is contained in the \code{cz.cuni.mff.ksi.jinfer.base.graphics} and \code{cz.cuni.mff.ksi.jinfer.base.graphics.icons} packages.

\subsection{Data structures}

Regular expression representation is contained in \code{cz.cuni.mff.ksi.jinfer.base.regexp} package. Most important class is of course \code{Regexp}, assisted by an enum of its type \code{RegexpType} and representation of its interval \code{RegexpInterval}.\\
XML node representation is described by interfaces in \code{cz.cuni.mff.ksi.jinfer.base.interfaces.nodes} package and more-less concrete implementations in \code{cz.cuni.mff.ksi.jinfer.base.objects.nodes} package.\\
Finite state automata representation is contained in the \cite{cz.cuni.mff.ksi.jinfer.base.automaton} package.\\
Refer to \ref{archdoc} to get an accurate description of all these representations.\\

Miscellaneous shared classes are contained in \code{cz.cuni.mff.ksi.jinfer.base.objects}. For example, \code{Input} is used to provide the \jmodule{Initial Grammar Generator} with input data. \code{Pair} is a generic class binding two object together in a \emph{pair}.

\subsection{Interfaces}

Apart from interfaces contained in already discussed \code{cz.cuni.mff.ksi.jinfer.base.interfaces.nodes} package, there is an important group of interfaces contained in \code{cz.cuni.mff.ksi.jinfer.base.interfaces.inference} package: the \emph{inference} interfaces. They come in two flavours: the actual inference interface, and its callback. For a comprehensive description of them and their interaction refer again to \ref{archdoc}.\\

There is one more package containing interfaces: \code{cz.cuni.mff.ksi.jinfer.base.interfaces}. There are a few groups of them.
\begin{itemize}
	\item Module lookup support: \code{NamedModule} and \code{UserModuleDescription}.
	\item Inference support: \code{Capabilities}.
	\item Service provider definitions: \code{Expander}, \code{Processor} and \code{RuleDisplayer}.
\end{itemize}

\subsection{Utility logic}

Useful logic for the whole framework is focused in the \code{cz.cuni.mff.ksi.jinfer.base.utils} package. A few highlights from here follow.
\begin{itemize}
	\item Testing whether a collection is empty: \code{BaseUtils.isEmpty()} - handles \code{null} and zero elements.
	\item Filtering a list based on a predicate: \code{BaseUtils.filter()}.
	\item Cloning a list N times in a row: \code{baseUtils.cloneList()} - e.g. from \emph{abc, 3 times} creates \emph{abcabcabc}.
	\item Deep cloning a grammar: \code{CloneHelper.cloneGrammar()}.
	\item Writing out a list separated by specified separator: \code{CollectionToString.colToString()} - accepts the list, operation to perform on each element and separator character. Separator is placed smartly, i.e. only between elements.
	\item XML element comparison while ignoring specified members: \code{EqualityUtils}.
	\item Module lookup/selection: \code{ModuleSelectionHelper}.
	\item Singleton class reporting on the currently running inference: \code{RunningProject}.
	\item Various utilities for JUnit tests: \code{TestUtils}.
	\item Topological sorting of grammar: \code{TopologicalSort}. 
\end{itemize}

\section{Data flow}

This is not an inference module and there is no real data flow in it. Refer to \ref{archdoc} to understand the inference process described by interfaces from \code{cz.cuni.mff.ksi.jinfer.base.interfaces.inference} package.

\section{Extensibility}

From one point of view, extensibility of \jmodule{Base} means creating service providers implementing inference or other interfaces. This is described in the documentation for respective modules that use these interfaces.\\
On the other hand, if there is a need to share data structures, interfaces or logic across multiple developed modules (for example between custom schema generator and simplifier), it is advised to create a module similar to \jmodule{Base} instead of changing it directly.

%\nocite{*}
\newpage
\bibliographystyle{alpha}
\bibliography{literature}

\end{document}
