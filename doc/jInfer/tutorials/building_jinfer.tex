\documentclass[a4paper,10pt,oneside]{article}
\usepackage{graphicx}
\usepackage{color}
\usepackage{url}
\usepackage{subfigure}
\usepackage[utf8]{inputenc}
\usepackage[T1]{fontenc}
\usepackage{tgpagella}
%\usepackage[scale=0.9]{tgcursor}
%\usepackage[scale=0.9]{tgheros}
\usepackage{xstring}
\usepackage{wrapfig}

\newcommand{\myscale}{0.74}
\newcommand{\vect}[1]{\boldsymbol{#1}}
\newcommand{\code}[1]{\texttt{\StrSubstitute{#1}{.}{.\.}}}
\def\.{\discretionary{}{}{}}
\newcommand{\jmodule}[1]{\texttt{\textit{#1}}}

\setlength{\hoffset}{-1in} %left margin will be 0, as hoffset is by default 1inch
\setlength{\voffset}{-1in} %analogous voffset
\setlength{\oddsidemargin}{1.5cm}
\setlength{\evensidemargin}{1.5cm}
\setlength{\topmargin}{1.5cm}
\setlength{\textheight}{24cm}
\setlength{\textwidth}{18cm}

\def\mftitle{jInfer: Building from sources}
\def\mfauthor{Michal Klempa, Mário Mikula, Robert Smetana, Michal Švirec, Matej Vitásek}
\def\mfadvisor{RNDr. Irena Mlýnková, Ph.D., Martin Nečaský, Ph.D.}
\def\mfplacedate{Praha, 2011}
\title{\bf\mftitle}
\author{\mfauthor \\ Advisors: \mfadvisor}
\date{\mfplacedate}

\ifx\pdfoutput\undefined\relax\else\pdfinfo{ /Title (\mftitle) /Author (\mfauthor) /Creator (PDFLaTeX) } \fi

\begin{document}

\maketitle




 
  
 



 \section*{Building jInfer from sources}
\par 
  Target audience: jInfer power users. Anyone who needs to build jInfer from sources.

\par 
  This document assumes that your computer is set to use jInfer, meaning you
    have correct version of NetBeans installed and you can run jInfer.

\subsection*{Getting jInfer sources}
\par 
There are two main ways of obtaining jInfer source files: from a version
  distribution, or from the Subversion repository. The first way might be more
  convenient and secure (version distributions are usually more stable than the
  bleeding edge code in repository), however you might prefer to build the
  latest code.

\par 
  The first way, obtaining sources from version distribution:

\begin{enumerate}\item 
 Go to \url{https://sourceforge.net/projects/jinfer/files/}.
  \item Download the latest jInfer sources (jInfer-X.Y-src.zip).
  \item Unpack the ZIP, you'll get a few directories. One of them will be \texttt{src}.
  \item Continue by building jInfer.
  \end{enumerate}
\par 
  Second way, checking out Subversion repository: you will need a
  Subversion client, such as the command line client or TortoiseSVN. In any case,
  \url{http://subversion.apache.org/}
might be a good place to start. To check out
  jInfer, use the following URL:
  \url{https://jinfer.svn.sourceforge.net/svnroot/jinfer}.
  After the checkout, continue by building jInfer.

\subsection*{Building from NetBeans}
\par 
  This is perhaps the easier, more user friendly way of building jInfer.
    If you however wish to automate the building process, you might want to
    try using Ant - in that case, see the next section.

\begin{enumerate}\item 
 Start NetBeans, select \textit{File} > \textit{Open Project...} from the main menu.
  \item In the dialog window, navigate to the root folder with jInfer sources.
    Its icon should look like two brown puzzle pieces.
  \item You should see the name ``jInfer'' in the dialog window.
  \item Make sure the \textit{Open Required Projects} check box is checked.
  \item You might want checking the \textit{Open as Main Project} checkbox.
  \item Click \textit{Open Project}. After a while, all jInfer projects should
    open and be visible in the \textit{Projects} window.
  \end{enumerate}\par At this moment, you might want to look around, play with the
  code a bit or apply some patches. If you want to try modified jInfer in a
  running NetBeans, select \textit{Run} from project jInfer (two brown puzzle
  pieces icon) context menu. If jInfer is the main project, clicking the
  green ``Play'' icon in the toolbar should work too.
  If you want to build jInfer, for example to test if your changes didn't break
  something, select \textit{Build} from the context menu. Sometimes it might be
  necessary to run \textit{Clean \& Build}, try this if it ``stops working''.
  If you want to run JUnit test, select \textit{Test} from the same context menu.
  If you now want to create your own distribution (= NBM modules), select
  \textit{Create NBMs} from the context menu. They will be created in
  \textit{build/updates} directory.
\subsection*{Building with Ant}
\par 
  Each jInfer module and jInfer suite itself has
  its own \texttt{build.xml} Ant build file. The most important one, for the
  whole jInfer suite, is located in the \texttt{src} folder.
  It is possible to use vanilla Ant from Apache, but perhaps safer to resort
  to the Ant bundled with NetBeans. This one should be located in \textit{NetBeans
    install directory} > \texttt{java/ant/bin/ant}.
  Whichever Ant is chosen, to build jInfer do the following on command line:

\begin{enumerate}\item 
 Change directory to jInfer's \texttt{src}.
  \item Run \texttt{ant [chosen target]}. Interesting targets include:
    \begin{itemize}\item 
        \texttt{clean}: removes any build artefacts.
      \item \texttt{build}: compiles and builds jInfer.
      \item \texttt{clean build}: performs a clean build, useful if something ``stops working''.
      \item \texttt{nbms}: creates a distribution by packaging jInfer modules in NBM files.
      \end{itemize}
  \end{enumerate}\par Chosen target may be empty, which should default to \texttt{build}.
 

 

\end{document}
