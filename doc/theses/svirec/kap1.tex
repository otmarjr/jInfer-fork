\chapter{Introduction}

Nowadays, one of the most largely used standards for information representation and data exchange on the Internet is the eXtensible Markup Language (XML) \cite{xml}. While XML as a markup language provides syntactic flexibility, the structure of an XML document is described with a so-called XML schema language. The two most popular XML schema languages proposed by W3C are Document Type Definition (DTD) \cite{xml} and XML Schema Definition (XSD) \cite{xsd1,xsd2,xsd3}. They support some kind of semantic content (e.g., keys and foreign keys), but for improvement of semantic expressiveness XML integrity constraints for XML \cite{ic} have been defined.

However, similarly to the problem of schemas also problem of detection of integrity constrains has two distinct aspects. First, if integrity constraints are not explicitly expressed, they need to be detected from the given set of data. Second, if integrity constraints are expressed, XML document may not be consistent with respect to them and this needs to be detected (checking the satisfaction of integrity constraint in an XML document). Moreover, detected XML data inconsistency needs to be repaired. We choose this very aspect of detection of integrity constraints along with the repair of the inconsistent XML document. Since several different classes of integrity constraints have been defined for XML, we choose functional dependency (FD) \cite{fd}, which is the most common semantic constraint used in relational databases.

The problem of checking the satisfaction of functional dependencies in an XML document is studied in \cite{satifFD}. Also algorithms computing a repair applied to an XML document such that functional dependencies will be satisfied again have been proposed \cite{RepAndConsistentAnswer, QueryXML, ImprovingXML}.

\section{Aim of the Thesis}

As we mention before, we focus on the detection of XML data inconsistency with respect to the functional dependency and consequential repairing this violation. The aim of the thesis is to propose an algorithm repairing XML functional dependency violations for a given XML document and a set of functional dependencies. It analyses recent approaches and discusses their advantages and disadvantages. The focus of the thesis is to incorporate the weight model and to involve the user into the process of finding and applying the repair and find out how this interaction helps. A part of this thesis is an experimental implementation of the proposed algorithm and its experimental evaluation.

\section{Structure of the Thesis}

Chapter 2 introduces basic definitions which are necessary for further chapters. Chapter 3 presents recent approaches of repairing XML functional dependency violations. In Chapter 4 is the main algorithm of this thesis proposed. Chapter 5 contains details of the experimental implementation. Experimental result are presented in Chapter 6. Finally, Chapter 7 contains a conclusion and several suggestions for future work.
