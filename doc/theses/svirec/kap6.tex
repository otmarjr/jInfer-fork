\chapter{Experimental Results}

In this chapter, experimental results of the \texttt{FDRepairer} module of the jInfer framework are presented. Set of documents and functional dependencies have been provided as an input and results from the former algorithm have been compared with results from \texttt{FDRepairer}. Documents in provided datasets are split into real-world and synthetic XML documents.

\section{Datasets}

All provided documents with functional dependencies and also repaired documents are placed on the enclosed CD in the directory \texttt{/datasets/}.

\subsection{Real-World Data}

The first real-world dataset originates from the XML data repository \\(\url{http://www.cs.washington.edu/research/xmldatasets/}), specifically the Course data derived from university websites. The course data consist, along with other information, of day of the week, time and place (building and room), where each course is situated. DTD of one of the dataset file is shown in Fig. \ref{courseDTD}. The consistency of data is evaluated against FD defined as: \emph{Two courses starting at the same date and time are each situated in a different place}.\\

\begin{figure}
    \begin{verbatim}
        <!ELEMENT root (course*)>
        <!ELEMENT course (reg_num,subj,crse,sect,title,units,
                            instructor,days,time,place)>
        <!ELEMENT reg_num (#PCDATA)>
        <!ELEMENT subj (#PCDATA)>
        <!ELEMENT crse (#PCDATA)>
        <!ELEMENT sect (#PCDATA)>
        <!ELEMENT title (#PCDATA)>
        <!ELEMENT units (#PCDATA)>
        <!ELEMENT instructor (#PCDATA)>
        <!ELEMENT days (#PCDATA)>
        <!ELEMENT time (start_time,end_time)>
        <!ELEMENT start_time (#PCDATA)>
        <!ELEMENT end_time (#PCDATA)>
        <!ELEMENT place (building,room)>
        <!ELEMENT building (#PCDATA)>
        <!ELEMENT room (#PCDATA)>
    \end{verbatim}
    \caption{DTD of reed.xml from course dataset}
    \label{courseDTD}
\end{figure}

The second dataset is a set of actors of IMDB database obtained from the Niagara data source (\url{http://www.cs.wisc.edu/niagara/data.html}). The dataset consists of a set of authors, where for each author a set of movies he played in is listed. DTD of this dataset is shown in Fig. \ref{actorsDTD}. For this dataset, we defined FD as follows: \emph{For each two authors, which played in the movie with the same name, the year of release of this movie must be the same}.

\begin{figure}[H]
    \begin{verbatim}
        <!ELEMENT W4F_DOC (Actor)>
        <!ELEMENT Actor (Name,Filmography)>
        <!ELEMENT Name (FirstName, LastName)>
        <!ELEMENT FirstName (#PCDATA)>
        <!ELEMENT LastName (#PCDATA)>
        <!ELEMENT Filmography (Movie)*>
        <!ELEMENT Movie (Title,Year)>
        <!ELEMENT Title (#PCDATA)>
        <!ELEMENT Year (#PCDATA)>
    \end{verbatim}
    \caption{DTD of actors.xml}
    \label{actorsDTD}
\end{figure}

\subsection{Synthetic Data}

Two synthetic datasets have been constructed, the former represents data introduced in Example \ref{example1} with FD presented in Example \ref{fdExample}. The latter is created according to Example 1 introduced in \cite{ImprovingXML}. DTD of the former dataset is identical with DTD defined in Example \ref{pathExample}. The latter datasets DTD is shown in Fig. \ref{synthDTD}.

\begin{figure}[H]
    \begin{verbatim}
        <!ELEMENT customers (country)>
        <!ELEMENT country (name,c_list)>
        <!ELEMENT name (#PCDATA)>
        <!ELEMENT c_list (customer)*>
        <!ELEMENT customer (no,phone,zip,city)>
        <!ELEMENT no (#PCDATA)>
        <!ELEMENT phone (#PCDATA)>
        <!ELEMENT zip (#PCDATA)>
        <!ELEMENT city (#PCDATA)>
    \end{verbatim}
    \caption{DTD of customers.xml}
    \label{synthDTD}
\end{figure}

\section{Algorithms Comparison}

For each dataset we perform the following. First, we run the original algorithm and collect from a repaired document informations about how many nodes have been modified with particular update operation. Moreover, we split the value modification operation into two types. The first is changing value to new one and the second one is copying the value from one node to another.

After that, we use our proposed algorithm with minimal repair candidate selection mode, and along with the data we collect for original algorithm, we also gather information on how many repair groups have been created before the first application of repair candidate and the total number of picked repair groups. We run our algorithm multiple times with different value of the coefficient $k$ and assignment of node weights $w$.\\

The first real-world data (course data) is represented by \texttt{reed.xml} and \texttt{wsu.xml}. For each document, we execute our algorithm with three different settings. First two use the default value of $k$ ($k=1.5$), for the last one, we set $k=0.005$. For the last two executions, we set for \texttt{reed.xml} (\texttt{wsu.xml}) the weight $w$ of nodes represented by XPath \texttt{/root/course/time/start\_time/text()} (resp. \texttt{/root/\discretionary{}{}{}course/time/\discretionary{}{}{}start/text()}) to 0.1.

The document \texttt{actors.xml} represents the second real-world dataset. Our algorithm is executed two times, where for both executions $k=1.5$ is set. Moreover, the second has weights of nodes represented by XPath \texttt{/W4F\_DOC/\discretionary{}{}{}Actor/Filmography/Movie/Year/text()} set to $w=0.2$.\\

Generated datasets represented by \texttt{bib.xml} and \texttt{customers.xml} documents have been both used as a input for our algorithm two times, where first execution was done with the default value of $k$. The second execution on \texttt{bib.xml} uses $k=0.1$ and on \texttt{customers.xml} uses $k=1.5$ and $w=0.2$ is set to all nodes represented by \texttt{/customers/country/c\_list/customer/city/text()}.

\section{Results}

All information gathered from real-world and synthetic datasets is shown in Tables \ref{realWorldTable} and \ref{syntheticTable}. The columns of both tables are defined as follows: \emph{RG} is the count of repair groups created before application of repair candidate, \emph{RG total} is the total count of picked repair groups, \emph{U} specifies the number of nodes marked as unreliable, \emph{NV} defines the number of nodes changing value to a new one and finally \emph{ChV} specifies the number of nodes with value copied from another node.\\


\begin{table}
    \begin{tabular}{| l | r | *{7}{c|}}
    \hline
    dataset & repairer & $k$ & $w$ & RG & RG total & U & NV & ChV\\ \hline
    \multirow{4}{*}{reed.xml} & old repairer & - & - & - & - & 0 & 780 & 0\\
    & \texttt{FDRepairer} & 1.5 & - & 218 & 195 & 0 & 195 & 0\\
    & \texttt{FDRepairer} & 1.5 & 0.1 & 218 & 195 & 0 & 195 & 0\\
    & \texttt{FDRepairer} & 0.005 & 0.1 & 218 & 195 & 5265 & 0 & 0\\ \hline
    \multirow{4}{*}{wsu.xml} & old repairer & - & - & - & - & 0 & 2612 & 0\\
    & \texttt{FDRepairer} & 1.5 & - & 3520 & 653 & 0 & 653 & 0\\
    & \texttt{FDRepairer} & 1.5 & 0.1 & 3520 & 653 & 0 & 653 & 0\\
    & \texttt{FDRepairer} & 0.005 & 0.1 & 3520 & 653 & 20833 & 0 & 0\\ \hline
    \multirow{3}{*}{actors.xml} & old repairer & - & - & - & - & 0 & 62 & 62\\
    & \texttt{FDRepairer} & 1.5 & - & 69 & 62 & 0 & 32 & 30\\
    & \texttt{FDRepairer} & 1.5 & 0.2 & 69 & 79 & 0 & 0 & 79\\ \hline
    \end{tabular}
\caption{Real-World datasets}
\label{realWorldTable}
\end{table}

From the information presented in these tables is apparent that our approach found for each dataset a repair that modifies less nodes than the original repair algorithm. One exception where our algorithm modifies more nodes is the case when we set value of $k$ near 0, which marks nodes as unreliable. With this setting we want to demonstrate that with our approach it is possible to mark nodes as unreliable instead of modifying their values. This is not possible with the original repairer, since it prefers node value modification to marking node as unreliable.

\begin{table}
    \begin{tabular}{| l | r | *{7}{c|}}
    \hline
    dataset & repairer & $k$ & $w$ & RG & RG total & U & NV & ChV\\ \hline
    \multirow{3}{*}{bib.xml} & old repairer & - & - & - & - & 0 & 913 & 0\\
    & \texttt{FDRepairer} & 1.5 & - & 1444 & 913 & 0 & 913 & 0\\
    & \texttt{FDRepairer} & 0.1 & - & 1444 & 913 & 5478 & 0 & 0\\ \hline
    \multirow{3}{*}{customers.xml} & old repairer & - & - & - & - & 0 & 99 & 99\\
    & \texttt{FDRepairer} & 1.5 & - & 232 & 136 & 0 & 72 & 64\\
    & \texttt{FDRepairer} & 1.5 & 0.2 & 232 & 361 & 0 & 0 & 361\\\hline
    \end{tabular}
\caption{Synthetic datasets}
\label{syntheticTable}
\end{table}

Setting the weight to particular nodes causes two different behaviour of the algorithm. First, setting the weights effects the choise of strategies used to repair the violations. In datasets \texttt{actors.xml} and \texttt{customers.xml} setting the custom weight to nodes causes the node values are changed to values of other nodes. Second causes modification of nodes defined by the path specifying the weights. In datasets \texttt{reed.xml} and \texttt{wsu.xml} setting the weights still changes the value of nodes to a newly generated one, however for \texttt{reed.xml} only values of nodes defined by \texttt{/root/course/time/start\_time/\discretionary{}{}{}text()} are modified and for \texttt{wsu.xml} are modified values defined by \texttt{root/\discretionary{}{}{}course/time/start/text()}.


Another interesting observation is that for almost all datasets, our algorithm picks less repair groups (from which picks repair candidate to apply to document) than were created before first pick. It means that one of the repair candidate repairs more than one FD violation and therefore less value modifications are needed. In the case where the number of picked repair groups outreach the number of firstly created ones, it means that some repair candidate introduces new violation which was consequently repaired in the next iteration of our algorithm.
