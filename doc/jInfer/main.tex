\documentclass[a4paper,10pt,oneside,twocolumn]{article}
\usepackage{epsfig}
\usepackage{color}
\usepackage{url}
\usepackage[utf8]{inputenc}
\usepackage[T1]{fontenc}
\usepackage{amsmath}
\usepackage{amssymb}
\usepackage{import}

\newcommand{\myscale}{0.74}
\newcommand{\vect}[1]{\boldsymbol{#1}}
\setlength{\hoffset}{-1in} %left margin will be 0, as hoffset is by default 1inch
\setlength{\voffset}{-1in} %analogous voffset
\setlength{\oddsidemargin}{1.5cm}
\setlength{\evensidemargin}{1.5cm}
\setlength{\topmargin}{1.5cm}
\setlength{\textheight}{26cm}
\setlength{\textwidth}{18cm}



\def\mftitle{jInfer - Framework for Automatic XML Schema Inference}
\def\mfauthor{Michal Klempa, Mário Mikula, Robert Smetana, Michal Švirec, Matej Vitásek}
\def\mfadvisor{RNDr. Irena Mlýnková, Ph.D., Martin Nečaský, Ph.D.}
\def\mfplacedate{Praha, 2011}
\title{\bf\mftitle}
\author{\mfauthor \\ Advisors: \mfadvisor}
\date{\mfplacedate}

\ifx\pdfoutput\undefined\relax\else\pdfinfo{ /Title (\mftitle) /Author (\mfauthor) /Creator (PDFLaTeX) } \fi

\begin{document}
\maketitle

\section*{Abstract}

jInfer is a framework for XML schema inference based on NetBeans platform.
Using jInfer, you can create DTD or XSD schema from your existing XML documents, schemas and queries.
With our modules, you can fine-tune the inference process and see it unfolding.
With your modules, you can do even more: export to exotic languages, try cutting-edge algorithms and more. 

\paragraph{keywords:} XML, schema


\section*{Introduction}

\section*{Tutorial}

\nocite{*}
\bibliographystyle{alpha}
\bibliography{literature}

\end{document}
