\chapter{Introduction}
The Extensible Markup Language (XML) \cite{Bray:08:EML} has become a popular standard for data representation and exchange. Its prevalence is based on its simplicity and, at the same time, its flexibility and expressive power.

With XML used to represent data themselves, it is often needed or convenient to somehow specify the structure of the represented data, their format, internal relations or restrictions, etc. In order to this need, several so-called XML schema languages (XML schemata) have been created. The most used of them are Document Type Definition (DTD) \cite{Bray:08:EML} and XML Schema (or XSD as XML Schema Definition) \cite{Walmsley:04:XSP, Thompson:04:XSP, Malhotra:04:XSP} proposed by the World Wide Web Consorcium (W3C).

Despite a conveyed encouragement to use an XML schema along with an XML data representation, in practise it is done sparsely. Commonly, XML documents are not assigned with their respective XML schema at all or the schema is outdated due to modifications done to a structure of data without updating the schema \cite{Mlynkova:2008:AAX:1494650.1495496}.

Recently in reaction to this situation, a significant number of approaches dealing with automatic construction of an XML schema has been proposed. The aim is to exploit a provided set of XML documents and infer an XML schema, so that the XML documents are valid against it. In addition, the inferred schema should be reasonable in views of human-readability, preciseness and conciseness. Most of published approaches to XML schema inference are of this type - the input is a set of XML documents. They are based on various ideas and can be classified by several aspects as discussed in Chapter \ref{chapter_analysis_of_recent_approaches}.

Besides the mentioned type, some approaches that utilize other or additional sources have been developed, for example \cite{Mlynkova:2009:IXS:1862681.1862693} and \cite{thesis_klempa}. If there are available sources like an out-dated XML schema, operations upon the XML data such as a set of XQuery \cite{w3c_xquery} queries, or other, these sources can be exploited to refine the process of inference. The refinement can be achieved in various aspects such as increasing the speed of the process, getting a more precise, more concise or more readable result or inference of statements about the data which cannot be (easily) extracted from the data themselves.

Recently, the main effort has been focused on a research of the approaches that utilize XML documents, and thus, there are only few approaches of the latter type (as also discussed in Chapter \ref{chapter_analysis_of_recent_approaches}), leaving a wide space for a possible future research and improvements.

\section{Aim of this Work}
Our aim is to perform a research on the problem of inference of an XML schema for the given set of XML data in a situation when we are provided also with a set of related operations (XML queries, XSLT scripts \cite{Clark:99:XTV} etc.).

Firstly, we analyze existing inference solutions in general and discuss their advantages and disadvantages. Then, we identify and discuss information that can be extracted from a given set of XML operations and how they can be exploited to achieve more precise and realistic XML schema.

The work results in a proposal of own approach involving the improvements, its implementation and suitable experiments that show its advantages.

\section{Structure of the Thesis}
The thesis is structured as follows.

In Chapter 2, we introduce XML technologies used in this thesis and we briefly explain their basic principles.

Chapter 3 contains summaries of several existing approaches of XML schema inference. One of them proposes an utilisation of XQuery queries, which interests us the most.

We made an analysis of XQuery technology from a view of XML schema inference, where we suggest several ideas on how XQuery queries can be exploited in the inference process. This is a content of Chapter 4.

In Chapter 5, we discuss some essential questions that we needed to decide before a development of an algorithm.

The algorithm itself is proposed in Chapter 6. And Chapter 7 describes how to combine results of the algorithm with existing grammar inferring methods of XML schema inference.

An implementation of the proposed algorithm using the jInfer framework is described in Chapter 8. Chapter 9 then discusses performed experiments and their outcomes. And, finally, Chapter 10 concludes.
